%%% introduction.tex ---

%% Author: dominikho@gmx.net
%% Version: $Id: introduction.tex,v 0.0 2009/02/23 15:09:18 dominikh Exp$

\chapter{Einleitung}
\section{Entstehung der Idee}
In dieser Facharbeit werde ich mich mit XML beschäftigen und
untersuchen, wie dieses auf einfache und effektive Art und Weise
mittels PHP dazu verwendet werden kann, ein elektronisches Kaufhaus,
im Folgenden als {\em Anwendung} bezeichnet, zu entwickeln. Dafür werde
ich XML ausführlich erklären und den Entwicklungsverlauf der Anwendung
veranschaulicht darstellen.

Diese Idee hat ihren Ursprung in der Frage, ob XML eine angebrachte
Alternative zu traditionellen Datenbanksystemen wie MySQL darstellen
kann, oder ob man unvertretbaren Problemen gegenüber stehen würde.

Diese Frage entstand hauptsächlich vor dem Hintergrund der
Entwicklungen des {\em Web 2.0}, welches die Idee von verteilten
Inhalten prägte. Eben diese machen es mehr denn je notwendig, dass
Informationen auf einfache Art und Weise, wie eben zum Beispiel per XML,
ausgetauscht werden können. Gerade deswegen wäre es Interessant, Daten
direkt per XML zu speichern und zu verarbeiten, um unnötige
Zwischenschritte zu vermeiden, wie sie notwendig wären, wenn die Daten
zum einen in einer Datenbank wie MySQL gespeichert wären, zu gleich
aber auch per XML ausgetauscht werden sollten.

Aus dieser Idee entsteht der zweite Gesichtspunkt dieser Facharbeit,
nämlich welche konzeptionellen Unterschiede es zwischen XML und
traditionellen Datenbanksystemen gibt. Hierfür werde ich die
Eigeschaften von XML, welche ich im ersten Teil erarbeiten werde,
denen von MySQL, stellvertretend für relationelle Datenbanksysteme,
gegenüberstellen.

\section{Abgrenzung}
Auch wenn diese Facharbeit die Entstehung eines kleinen elektronischen
Kaufhauses zum Thema haben wird, werde ich nicht auf alle Aspekte der
Entwicklung eingehen. Das heißt, dass ich nur jenes beschreiben werde,
was mittelbar oder unmittelbar mit XML bzw. PHP in Verbindung mit XML
zu tun hat. Ich werde nicht auf jene Aspekte eingehen, welche rein auf
Gestaltung der Webseite (sprich HTML und CSS) abzielen, da dies den
Rahmen dieser Facharbeit sprengen würde.

Des Weiteren werde ich darauf verzichten, einen Warenkorb zu
entwickeln und stattdessen auf {\em mail()} zurückgreifen, um
Produktbestellungen zu übermitteln. Auch dies dient dazu, den Rahmen
der Facharbeit nicht zu sprengen.

%%% Local Variables:
%%% mode: latex
%%% TeX-master: "index"
%%% End:
