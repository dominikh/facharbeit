%%% index.tex ---

%% Author: dominikho@gmx.net
%% Version: $Id: index.tex,v 0.0 2008/10/20 19:05:25 dominikh Exp$

\documentclass[12pt,oneside,ngerman,a4paper,bibgerm]{scrreprt}
\usepackage[a4paper,left=4cm,right=2cm,top=2cm,bottom=2cm]{geometry}
\usepackage[T1]{fontenc}
\usepackage[utf8]{inputenc}
\usepackage[ngerman]{babel}
\usepackage{scrpage2}
\pagestyle{scrheadings}
\usepackage[numbers,sort&compress]{natbib}
\usepackage{bibgerm}
\usepackage{url}
\usepackage{setspace}
\usepackage{tabularx}
\usepackage{remreset}
\usepackage{listings}
\usepackage{color}
\usepackage{graphicx}
\usepackage{wrapfig}
\usepackage{fancybox}
\usepackage{babel}
\usepackage[babel]{csquotes}
\usepackage{wallpaper}

\usepackage{enumitem}

\newcommand{\myquote}[2]{
  \begin{quote}
    \enquote{#1}#2
  \end{quote}
}

\definecolor{grey}{gray}{0.5}

\newcommand{\signature}[1]{
  \vspace*{\bigskipamount}
  \hfill
  \begin{tabularx}{0.95\linewidth}{Xr}
    \\
    \\
    \\
    \cline{2-2}
    & \makebox[4cm][r]#1
  \end{tabularx}
  \vspace*{\bigskipamount}
}

\newcommand{\BibTeX}{
  $\mathrm{B{\scriptstyle{IB}} \! T\!_{\displaystyle E} \! X}$
}

\newenvironment{VarDescription}[1]%
  {\begin{list}{}{\renewcommand{\makelabel}[1]{\textbf{##1:}\hfil}%
    \settowidth{\labelwidth}{\textbf{#1:}}%
    \setlength{\leftmargin}{\labelwidth}\addtolength{\leftmargin}{\labelsep}}}%
  {\end{list}}

\makeatletter% --> De-TeX-FAQ
\renewcommand*{\lstlistoflistings}{%
  \begingroup
    \if@twocolumn
      \@restonecoltrue\onecolumn
    \else
      \@restonecolfalse
    \fi
    \lol@heading
    \setlength{\parskip}{\z@}%
    \setlength{\parindent}{\z@}%
    \setlength{\parfillskip}{\z@ \@plus 1fil}%
    \@starttoc{lol}%
    \if@restonecol\twocolumn\fi
  \endgroup
}
\makeatother% --> \makeatletter

\renewcommand*{\chapterpagestyle}{scrheadings}
\renewcommand{\chaptermark}[1]{%
\markboth{#1}{}}
\chead{-\pagemark-} \ihead{} \ohead{} \cfoot{}

\makeatletter
\@removefromreset{footnote}{chapter}
\makeatother

%% \usepackage[debugshow,final]{graphics}

%% \revision$Header: /home/dominikh/bibtex/index.tex,v 0.0 2008/10/20 19:05:26 dominikh Exp$

\author{Dominik Honnef}
\title{Elektronisches Kaufhaus mit XML}
\date{2008/09}

\begin{document}
\onehalfspacing
\clearpage
\begingroup
\ThisCenterWallPaper{0.75}{images/db}
\renewcommand*{\chapterpagestyle}{empty}
\pagestyle{empty}
\vspace*{\fill}
\begin{center}
  \textbf{\huge{Elektronisches Kaufhaus mit XML}} \\
  \bigskip
  \small{Dominik Honnef} \\
  \bigskip
  \small{2008/09}
\end{center}
\vspace*{\fill}
\begin{tabular}{|l|r|}
  \hline
  \textbf{Schule} & Gymnasium im Loekamp \\
  \hline
  \textbf{Kurs} & If1 \\
  \hline
  \textbf{Schuljahr} & 12 \\
  \hline
  \textbf{Fachlehrer} & Herr Wlost \\
  \hline
\end{tabular}
\clearpage
\tableofcontents{}
\clearpage
\endgroup
\newpage
%%% introduction.tex ---

%% Author: dominikho@gmx.net
%% Version: $Id: introduction.tex,v 0.0 2009/02/23 15:09:18 dominikh Exp$
\chapter{Einleitung}
\section{Entstehung der Idee}
In dieser Facharbeit werde ich mich mit XML beschäftigen und
untersuchen, wie dieses auf einfache und effektive Art und Weise
mittels PHP dazu verwendet werden kann, ein elektronisches Kaufhaus,
im Folgenden als {\em Anwendung} bezeichnet, zu entwickeln. Dafür werde
ich XML ausführlich erklären und den Entwicklungsverlauf der Anwendung
veranschaulicht darstellen.

Diese Idee hat ihren Ursprung in der Frage, ob XML eine angebrachte
Alternative zu traditionellen Datenbanksystemen wie MySQL darstellen
kann, oder ob man unvertretbaren Problemen gegenüber stehen würde.

Diese Frage entstand hauptsächlich vor dem Hintergrund der
Entwicklungen des {\em Web 2.0}, welches die Idee von verteilten
Inhalten prägte. Eben diese machen es mehr denn je notwendig, dass
Informationen auf einfache Art und Weise, wie eben zum Beispiel per XML,
ausgetauscht werden können. Gerade deswegen wäre es interessant, Daten
direkt per XML zu speichern und zu verarbeiten, um unnötige
Zwischenschritte zu vermeiden, wie sie notwendig wären, wenn die Daten
zum einen in einer Datenbank wie MySQL gespeichert wären, zu gleich
aber auch per XML ausgetauscht werden sollten.

Aus dieser Idee entsteht der zweite Gesichtspunkt dieser Facharbeit,
nämlich welche konzeptionellen Unterschiede es zwischen XML und
traditionellen Datenbanksystemen gibt. Hierfür werde ich die
Eigeschaften von XML, welche ich im ersten Teil erarbeiten werde,
denen von MySQL, stellvertretend für relationelle Datenbanksysteme,
gegenüberstellen.

\section{Abgrenzung}
Auch wenn diese Facharbeit die Entstehung eines kleinen elektronischen
Kaufhauses zum Thema haben wird, werde ich nicht auf alle Aspekte der
Entwicklung eingehen. Das heißt, dass ich nur jenes beschreiben werde,
was mittelbar oder unmittelbar mit XML bzw. PHP in Verbindung mit XML
zu tun hat. Ich werde nicht auf jene Aspekte eingehen, welche rein auf
Gestaltung der Webseite (sprich HTML und CSS) abzielen, da dies den
Rahmen dieser Facharbeit sprengen würde.

Des Weiteren werde ich darauf verzichten, einen Warenkorb zu
entwickeln und stattdessen auf {\em mail()} zurückgreifen, um
Produktbestellungen zu übermitteln. Auch dies dient dazu, den Rahmen
der Facharbeit nicht zu sprengen.

%%% Local Variables:
%%% mode: latex
%%% TeX-master: "index"
%%% End:

\chapter{XML}
\section{Beschreibung}
XML, kurz für {\em Extensible Markup Language}, ist eine
Auszeichnungssprache zur Darstellung und zum Austausch von
\enquote{hierarchisch strukturierten Daten}\cite{wiki:de:xml}. Das
{\em Extensible Language} rührt daher, dass der Entwickler selber
definieren kann beziehungsweise muss, welche {\em Mark-up Elemente} in
dem jeweiligen Dokument erlaubt sind. XML stellt somit nur eine
Metasprache für das festlegen einer eigenen Sprache dar. Wie dieses
Festlegen geschieht, wird im spaeteren Verlauf dargestellt.

\section{Aufbau}
\lstset{language=XML,
  numbers=left,
  stringstyle=\ttfamily,
  frame=box,
  rulesepcolor=\color{grey},
  basicstyle=\small
}
\lstset{caption=XML-Dokument; Quelle: http://de.wikipedia.org/wiki/XML\cite{wiki:de:xml}}
\lstinputlisting{code/examples/xml.xml}

\section{Schemasprachen}
\subsection{DTD}
\subsection{XML-Schema}
\chapter{Unsere Anwendung}
\section{Vorüberlegungen}
Bevor ich mit der Entwicklung der Anwendung beginne, muss ich einige
Vorüberlegungen machen, um nicht hinterher größere Teile wegen
Fehlkonzeptionen ändern zu müssen.

\begin{enumerate}
   \item Welche API werde ich zur Ansprache der XML Datei verwenden?
  Rückblickend auf auf Abschnitt \ref{xml-php} (Seite \pageref{xml-php})
  fällt die Entscheidung für SimpleXML nicht schwer.

   \item Da sowohl DOM als auch SimpleXML Probleme mit komplexeren Dokumenten
  haben, stellt sich die Frage, ob ich mit einer einzigen Datei für alle Produkte
  arbeiten werde, oder ob ich die Dateien aufteile und dynamisch zusammensetze.

  In Anbetracht der Tatsache, dass unsere Anwendung nur über eine kleine
  Anzahl von Produkten verfügen wird, werde ich auf das dynamische Zusammensetzen
  verzichten.
   \item Da SimpleXML bereits Objekte zurückliefert und unsere Objekte für Kategorien
  und Produkte nicht viel mehr können müssen, als Werte zurückzuliefern (und ggf. zu setzen)
  werde ich keine eigenen Klassen schreiben.

   \item Die Konfigurationsdatei, welche voraussichtlich ohnehin nur die Emailadresse
  beinhalten wird, wird per XML realisiert werden.

   \item Anstatt einer ausgefeilten Templateengine wie Smarty\footnote{http://www.smarty.net/} werde ich mit
  einfachen {\em includes} arbeiten

   \item Auch die Ordnerstruktur muss gut durchdacht sein. Ich werde folgende verwenden:
  \begin{VarDescription}{}
     \item[/] Wird  alle Unterordner und die index.php beinhalten, welche
    vom Browser aufgerufen wird

     \item[/etc/] Wird Konfigurationen, Datenbanken und Bilder
    beinhalten

     \item[/lib/] Wird alle Klassen beinhalten

     \item[/templates/] Wird alle Seiten der Anwendung beinhalten
  \end{VarDescription}
\end{enumerate}
\section{Vorbereitungen}
Bis auf das Anlegen der Ordner etc. werde ich keine Vorbereitungen zu
treffen haben, da ich bereits eine vollständige Entwicklungsumgebung
für Webanwendungen, sprich Apache, PHP, Mailserver und einen
angebrachten Editor, installiert habe.

\section{Entwicklung}

%%% Local Variables:
%%% mode: latex
%%% TeX-master: "index"
%%% End:

%%%% ##########################################################################
% \addcontentsline{toc}{chapter}{Listings}
% \lstlistoflistings
%%%% ##########################################################################
% \listoffigures
%\liststotoc
%%%% ##########################################################################
\bibliographystyle{plaindin}
\bibliography{db}
\addcontentsline{toc}{chapter}{\bibname}

%% remove these when using real \cite's %%
\nocite{pdf:xslt}
\nocite{pdf:xquery}
\nocite{pdf:xml}
\nocite{pdf:xslfo}
\nocite{wiki:de:xml}
\nocite{book:phpmysql}
%% ------------------------------------ %%
%%% addendum.tex ---

%% Author: dominikho@gmx.net
%% Version: $Id: addendum.tex,v 0.0 2009/02/26 12:57:35 dominikh Exp$

%%\revision$Header: /home/dominikh/projects/facharbeit/addendum.tex,v 0.0 2009/02/26 12:57:35 dominikh Exp$

\chapter*{Anhang}
\addcontentsline{toc}{chapter}{Anhang}
\section*{Liste aller Dateien}
\addcontentsline{toc}{section}{Liste aller Dateien}
\subsection*{Dateien zum Setzen der Facharbeit}
\addcontentsline{toc}{subsection}{Dateien zum Setzen der Facharbeit}
\begin{VarDescription}{images/background.png}
   \item[Makefile] Das Makefile zum Kompilieren des Dokuments
   \item[addendum.tex] Der \LaTeX-Quelltext des Anhangs
   \item[db.bib] Die \BibTeX Datenbank aller verwendbaren Quellen
   \item[images/background.png] Das Hintergrundbild des Titelblatts
   \item[index.pdf] Dieses Dokument
   \item[index.tex] Das Grundgerüst dieses Dokuments
   \item[introduction.tex] Der \LaTeX-Quelltext der Einleitung
   \item[plaindin.bst] Style für \BibTeX zum Setzen von Literaturverzeichnissen nach DIN 1505\footnote{http://de.wikipedia.org/wiki/DIN\_1505-2}
   \item[application.tex] Der \LaTeX-Quelltext der Wiedergabe der Entwicklung der Applikation
   \item[xml.tex] Der \LaTeX-Quelltext der Beschreibungen zu XML
\end{VarDescription}

\subsection*{Verwendete Bücher}
\addcontentsline{toc}{subsection}{Verwendete Bücher}
{\small Ausgehend vom Verzeichnis {\em books/}}
\begin{VarDescription}{20\_XSLT\_2df3rfy.pdf}
   \item[20\_XSLT\_2df3rfy.pdf] XSLT\cite{pdf:xslt}
   \item[XML mit Java.pdf] XML mit Java\cite{pdf:xml}
   \item[XQuery.pdf] XQuery\cite{pdf:xquery}
   \item[XSL-FO.pdf] XSL-FO Praxis\cite{pdf:xslfo}
\end{VarDescription}

\subsection*{Verwendete Webseiten}
\addcontentsline{toc}{subsection}{Verwendete Bücher}
{\small Ausgehend vom Verzeichnis {\em websites/}}
\begin{VarDescription}{wiki-de-xpath}
   \item[wiki-de-dom/] Document Object Model\cite{wiki:de:dom}
   \item[wiki-de-web20/] Web 2.0\cite{wiki:de:web20}
   \item[wiki-de-xml/] Extensible Markup Language\cite{wiki:de:xml}
   \item[wiki-de-xpath/] XPath\cite{wiki:de:xpath}
   \item[www-ibm-xml/] XML for PHP developers, Part 1: The 15-minute PHP-with-XML starter\cite{www:ibm:xml}
\end{VarDescription}

\subsection*{Die Anwendung}
\addcontentsline{toc}{subsection}{Die Anwendung}
{\small Ausgehend vom Verzeichnis {\em code/shop/}}
\begin{VarDescription}{etc/db.xml}
   \item[etc/db.xml] Die Datenbank mit allen Produkten
   \item[etc/config.xml] Die Konfigurationsdatei
   \item[etc/images/] Alle Produktbilder
   \item[index.php] Die Hauptdatei der Applikation
   \item[lib/templateengine.php] Die Templateengine
   \item[lib/controllers/category.php] Die Verwaltung der Anzeige einer Kategorie
   \item[lib/controllers/order.php] Die Verwaltung des Bestellformulars
   \item[templates/] Die einzelnen Unterseiten
\end{VarDescription}

%%% Local Variables:
%%% mode: latex
%%% TeX-master: "index"
%%% End:

\chapter*{Erklärung}
\addcontentsline{toc}{chapter}{Erklärung}
"`Ich erkläre, dass ich die Facharbeit ohne fremde
Hilfe angefertigt und nur die
im Literaturverzeichnis angeführten Quellen und Hilfsmittel benutzt habe."' \\
\signature{Dominik Honnef}

"`Ich erkläre mich mit dem Umstand einverstanden, dass diese
Facharbeit archiviert und für andere Schüler frei zugänglich gemacht
wird. \textbf{Des Weiteren ist es erwünscht, dass, falls möglich, auch die
erreichte Note mit verzeichnet wird.}"' \\
\signature{Dominik Honnef}
\end{document}
%%% Local Variables:
%%% mode: latex
%%% TeX-master: t
%%% End:
