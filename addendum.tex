%%% addendum.tex ---

%% Author: dominikho@gmx.net
%% Version: $Id: addendum.tex,v 0.0 2009/02/26 12:57:35 dominikh Exp$

%%\revision$Header: /home/dominikh/projects/facharbeit/addendum.tex,v 0.0 2009/02/26 12:57:35 dominikh Exp$

\chapter*{Anhang}
\addcontentsline{toc}{chapter}{Anhang}
\section*{Liste aller Dateien}
\addcontentsline{toc}{section}{Liste aller Dateien}
\subsection*{Dateien zum Setzen der Facharbeit}
\addcontentsline{toc}{subsection}{Dateien zum Setzen der Facharbeit}
\begin{VarDescription}{images/background.png}
   \item[db.bib] Die \BibTeX Datenbank aller verwendbaren Quellen
   \item[index.tex] Das Grundgerüst dieses Dokuments
   \item[addendum.tex] Der \LaTeX-Quelltext des Anhangs
   \item[index.pdf] Dieses Dokument
   \item[introduction.tex] Der \LaTeX-Quelltext der Einleitung
   \item[Makefile] Das Makefile zum Kompilieren des Dokuments
   \item[plaindin.bst] Style für \BibTeX zum Setzen von Literaturverzeichnissen nach DIN 1505
   \item[images/background.png] Das Hintergrundbild des Titelblatts
\end{VarDescription}

\subsection*{Verwendete Bücher}
\addcontentsline{toc}{subsection}{Verwendete Bücher}
{\small Ausgehend vom Verzeichnis {\em books/}}
\begin{VarDescription}{20\_XSLT\_2df3rfy.pdf}
   \item[20\_XSLT\_2df3rfy.pdf] XSLT\cite{pdf:xslt}
   \item[XML mit Java.pdf] XML mit Java\cite{pdf:xml}
   \item[XQuery.pdf] XQuery\cite{pdf:xquery}
   \item[XSL-FO.pdf] XSL-FO Praxis\cite{pdf:xslfo}
\end{VarDescription}

\subsection*{Die Anwendung}
\addcontentsline{toc}{subsection}{Die Anwendung}
{\small Ausgehend vom Verzeichnis {\em code/shop/}}
\begin{VarDescription}{etc/db.xml}
  \item[etc/db.xml] Die Datenbank mit allen Produkten
\end{VarDescription}

%%% Local Variables:
%%% mode: latex
%%% TeX-master: "index"
%%% End:
