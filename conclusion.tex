\chapter{Fazit}
Abschließend gilt es die Frage zu beantworten, ob XML einen adequaten
Ersatz für MySQL und Konsorten darstellen kann. Ich habe zwar schon in
Abschnitt \ref{relational_databases} festgestellt, dass XML keine
Gemeinsamkeiten mit relationalen Datenbanken aufweist und auch in
vielen Aspekten unterlegen ist.

Und auch wenn es mir beim Entwickeln der Anwendung überaus leicht
gefallen ist, die XML Daten zu verarbeiten und anzuzeigen, so glaube
ich doch, dass es bei einem größeren Maßstab, wie bei einem echten
Webshop, schnell zu Problemen kommen würde.

Dennoch war es interessant, einen alternativen Ansatz auszuprobieren,
vor allem da das Wissen über XML in anderen Anwendungen von Vorteil
sein kann, eben da, wo es darauf ankommt, kleine Mengen an Daten
schnell und einfach auszutauschen und zu verarbeiten.

Leider war es mir im Verlauf dieser Facharbeit nicht möglich, auf
einige interessante Aspekte einzugehen. So mag zu allererst
aufgefallen sein, dass es keine Möglichkeit gibt, die Kategorien oder
Produkte in meiner Anwendung über ein Webinterface zu editieren. Dies
liegt nicht daran, dass es nicht möglich oder zu kompliziert wäre, im
Gegenteil, SimpleXML bietet sehr einfache Methoden hierfür.
Stattdessen fehlte mir schlussendlich einfach die Zeit. Des Weiteren
gibt es noch viele interessante Themen wie XSLT und XQuery, welche es
durchaus werd sind, genauer betrachtet zu werden. Aber dies würde
leider bereits den Umfang einer eigenständigen Facharbeit einfordern
und könnte nicht in einem Kapitel untergebracht werden.

%%% Local Variables:
%%% mode: latex
%%% TeX-master: "index"
%%% End:
