\chapter{XML}
\section{Beschreibung}
XML, kurz für {\em Extensible Markup Language}, ist eine
Auszeichnungssprache zur Darstellung und zum Austausch von
\enquote{hierarchisch strukturierten Daten}\cite{wiki:de:xml}.  Das {\em
Extensible Language} rührt daher, dass der Entwickler selber
definieren kann beziehungsweise muss, welche {\em Mark-up Elemente} in
dem jeweiligen Dokument erlaubt sind.  XML stellt somit nur eine
Metasprache für das festlegen einer eigenen Sprache dar. Wie dieses
Festlegen geschieht, wird im spaeteren Verlauf dargestellt.

\section{Aufbau}
\lstset{language=XML,
  numbers=left,
  stringstyle=\ttfamily,
  frame=box,
  rulesepcolor=\color{grey},
  basicstyle=\small
}
\lstset{caption=XML-Dokument; Quelle: http://de.wikipedia.org/wiki/XML\cite{wiki:de:xml}}
\lstinputlisting{code/examples/xml.xml}

\section{Schemasprachen}
\subsection{DTD}
\subsection{XML-Schema}