\chapter{Unsere Anwendung}
\section{Vorüberlegungen}
Bevor ich mit der Entwicklung der Anwendung beginne, muss ich einige
Vorüberlegungen machen, um nicht hinterher größere Teile wegen
Fehlkonzeptionen ändern zu müssen.

\begin{enumerate}
   \item Welche API werde ich zur Ansprache der XML Datei verwenden?
  Rückblickend auf auf Abschnitt \ref{xml-php} (Seite \pageref{xml-php})
  fällt die Entscheidung für SimpleXML nicht schwer.

   \item Da sowohl DOM als auch SimpleXML Probleme mit komplexeren Dokumenten
  haben, stellt sich die Frage, ob ich mit einer einzigen Datei für alle Produkte
  arbeiten werde, oder ob ich die Dateien aufteile und dynamisch zusammensetze.

  In Anbetracht der Tatsache, dass unsere Anwendung nur über eine kleine
  Anzahl von Produkten verfügen wird, werde ich auf das dynamische Zusammensetzen
  verzichten.
   \item Da SimpleXML bereits Objekte zurückliefert und unsere Objekte für Kategorien
  und Produkte nicht viel mehr können müssen, als Werte zurückzuliefern (und ggf. zu setzen)
  werde ich keine eigenen Klassen schreiben.

   \item Die Konfigurationsdatei, welche voraussichtlich ohnehin nur die Emailadresse
  beinhalten wird, wird per XML realisiert werden.

   \item Anstatt einer ausgefeilten Templateengine wie Smarty\footnote{http://www.smarty.net/} werde ich mit
  einfachen {\em includes} arbeiten

   \item Auch die Ordnerstruktur muss gut durchdacht sein. Ich werde folgende verwenden:
  \begin{VarDescription}{}
     \item[/] Wird  alle Unterordner und die index.php beinhalten, welche
    vom Browser aufgerufen wird

     \item[/etc/] Wird Konfigurationen, Datenbanken und Bilder
    beinhalten

     \item[/lib/] Wird alle Klassen beinhalten

     \item[/templates/] Wird alle Seiten der Anwendung beinhalten
  \end{VarDescription}
\end{enumerate}
\section{Vorbereitungen}
Bis auf das Anlegen der Ordner etc. werde ich keine Vorbereitungen zu
treffen haben, da ich bereits eine vollständige Entwicklungsumgebung
für Webanwendungen, sprich Apache, PHP, Mailserver und einen
angebrachten Editor, installiert habe.

\section{Installationshinweise}
Die Installation der Anwendung setzt einen funktionieren Webserver
(wie z.\,B. Apache) und PHP 5 voraus, wie es zum Beispiel im XAMPP
Paket zu finden ist. Des Weiteren ist es von Vorteil, einen
funktionieren Mailserver zu haben, damit die Bestellungen versendet
werden können. Sollte kein Mailserver vorhanden sein, so kann folgende
Zeile in der Datei {\em lib/controllers/order.php} entkommentiert
werden, um die Email stattdessen direkt im Browser zu sehen:
\lstinline|// echo '<pre>' . $message . '</pre>';|


\section{Entwicklung}
Als erstes entwickle ich eine minimalistische Klasse zur Verwaltung
der einzelnen Seiten, welche sich im Groben an Smarty orientiert. Sie
bietet somit eine {\em assign} Methode, mit welcher ich den Seiten
Variablen übergeben kann und eine {\em display} Methode, welche die
Seite darstellt. Des Weiteren besitzt die Klasse eine Methode {\em
  get}, mit welcher die Seiten auf die Variablen zugreifen können.
Verwenden werde ich diese Klasse so, dass sie die Hauptdatei der Seite
({\em templates/index.php} lädt, welche wiederum per {\em include} und
{\em get} die gewünschte Unterseite einbindet. Somit muss das
Grundlayout der Seite nur ein mal festgelegt werden.

Als nächstes schreibe ich die {\em index.php}, welche vom Browser
aufgerufen werden wird. Diese initialisiert eine Instanz der vorher
angelegten Klasse, lädt die Datenbank und überprüft, welche Unterseite
aufgerufen werden soll. Dafür haben wir ein Array, welches die
erlaubten Seitennamen beinhaltet. Sollte die gewünschte Seite nicht in
diesem vorhanden sein, wird eine Fehlermeldung ausgegeben. Des
Weiteren überprüfen wir, ob für die gewünschte Seite ein Controller
existiert, welcher weitergehende Aktionen durchführen wird, wie z.\,B.
im Falle des Bestellformulars zu überprüfen, ob alle benötigten
Informationen übergeben wurden.

Nach dem dieses Grundgerüst steht, werde ich die einzelnen Seiten
erstellen. Hierfür schreibe ich ein simples XHTML-Grundgerüst und
formatiere dieses mit CSS. Etwaige Variablen, die die Seite von der
Klasse benötigt, werden mittels PHP-Blöcken eingebunden. Beim
Schreiben der Seiten habe ich besonderes Augenmerk darauf gelegt,
semantisches und valides Markup zu schreiben. Für die Übersicht der
Kategorien und Produkte benutze ich zudem Schleifen, welche das
Markup dynamisch generieren.

Abschließend fehlen nur noch zwei Controller, namentlich die für die
Übersicht einer einzelnen Kategorie ({\em
  lib/controllers/category.php}) und die, welche das Bestellformular
({\em lib/controllers/order.php}) verwaltet. In dem Controller für die
Kategorie wird per XPath die jenige Kategorie ausgewählt, die der
Anwender vorher in der Übersicht der Kategorien ausgewählt hat. Dies
wird gezielt in einem Controller und nicht im Template gemacht, da
dies Anwendungslogik und nicht Ausgabe ist.

Der andere Controller hat einiges mehr an Arbeit. Dieser muss zum
einen kontrollieren, ob der Anwender bereits alle nötigen
Informationen angegeben hat, das heißt Namen, Anschrift, Emailadresse,
ob er nicht die URL per Hand in solcher Art und Weise manipuliert hat,
dass ein weiteres Arbeiten unmöglich wäre, und falls alle
Informationen angegeben wurden, muss er eine passende Email
konstruieren und uns zuschicken. Als kleine Hilfe für den Anwender
werden bereits eingegebene Daten wieder im Formular angezeigt, damit
er nicht alles von vorne eingeben muss.
%%% Local Variables:
%%% mode: latex
%%% TeX-master: "index"
%%% End:
