\chapter{Unsere Anwendung}
\section{Vorüberlegungen}
Bevor ich mit der Entwicklung der Anwendung beginne, muss ich einige
Vorüberlegungen machen, um nicht hinterher größere Teile wegen
Fehlkonzeptionen ändern zu müssen.

\begin{enumerate}
   \item Welche API werde ich zur Ansprache der XML Datei verwenden?
  Rückblickend auf auf Abschnitt \ref{xml-php} (Seite \pageref{xml-php})
  fällt die Entscheidung für SimpleXML nicht schwer.

   \item Da sowohl DOM als auch SimpleXML Probleme mit komplexeren Dokumenten
  haben, stellt sich die Frage, ob ich mit einer einzigen Datei für alle Produkte
  arbeiten werde, oder ob ich die Dateien aufteile und dynamisch zusammensetze.

  In Anbetracht der Tatsache, dass unsere Anwendung nur über eine kleine
  Anzahl von Produkten verfügen wird, werde ich auf das dynamische Zusammensetzen
  verzichten.
   \item Da SimpleXML bereits Objekte zurückliefert und unsere Objekte für Kategorien
  und Produkte nicht viel mehr können müssen, als Werte zurückzuliefern (und ggf. zu setzen)
  werde ich keine eigenen Klassen schreiben.

   \item Die Konfigurationsdatei, welche voraussichtlich ohnehin nur die Emailadresse
  beinhalten wird, wird per XML realisiert werden.

   \item Anstatt einer ausgefeilten Templateengine wie Smarty\footnote{http://www.smarty.net/} werde ich mit
  einfachen {\em includes} arbeiten

   \item Auch die Ordnerstruktur muss gut durchdacht sein. Ich werde folgende verwenden:
  \begin{VarDescription}{}
     \item[/] Wird  alle Unterordner und die index.php beinhalten, welche
    vom Browser aufgerufen wird

     \item[/etc/] Wird Konfigurationen, Datenbanken und Bilder
    beinhalten

     \item[/lib/] Wird alle Klassen beinhalten

     \item[/templates/] Wird alle Seiten der Anwendung beinhalten
  \end{VarDescription}
\end{enumerate}
\section{Vorbereitungen}
Bis auf das Anlegen der Ordner etc. werde ich keine Vorbereitungen zu
treffen haben, da ich bereits eine vollständige Entwicklungsumgebung
für Webanwendungen, sprich Apache, PHP, Mailserver und einen
angebrachten Editor, installiert habe.

\section{Entwicklung}

%%% Local Variables:
%%% mode: latex
%%% TeX-master: "index"
%%% End:
